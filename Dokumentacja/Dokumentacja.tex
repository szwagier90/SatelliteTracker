%\documentclass[titlepage]{article}
\documentclass[titlepage]{mwart}

\usepackage{microtype}
\DisableLigatures{encoding = *, family = *}

\usepackage[utf8]{inputenc}
\usepackage{polski}
\usepackage[polish]{babel}

\begin{document}
\title{Mikroprocesorowy Sterownik Śledzący Anteny Satelitarne}
\author{Paweł Szwagierek\\
Wydział Elektroniki,\\
Politechnika Wrocławska,\\
Nr Albumu: 184763\\
\texttt{184763@student.pwr.wroc.pl}
}
\date{\today}
\maketitle
\tableofcontents
\clearpage

\section{Wstęp}
***Celem ***

\section{Potrzebne oprogramowanie}
***fffisdasd***

Dokumentacja przygotowana w \LaTeXe

\section{Przygotowanie projektu w programie Keil}
Pierwszym ważnym krokiem jest stworzenie kompilującego się pustego projektu zawierającego wszystkie potrzebne biblioteki. Po uruchomieniu programu zobaczymy puste okna.

\section{Wnioski oraz podsumowanie}
***WWW***
\section{Bibliografia}
\begin{thebibliography}{9}

%The \bibitem is to start a new reference. Ensure that the cite_key is
%unique. You don't need to put each element on a new line, but I did
%simply for readability.

\bibitem{lamport94}
Leslie Lamport,
\emph{\LaTeX: A Document Preparation System}.
Addison Wesley, Massachusetts,
2nd Edition,
1994.

\end{thebibliography} %Must end the environment

\end{document} %End of document.