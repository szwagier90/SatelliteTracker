%\documentclass[12pt,a4paper]{article}
\documentclass[12pt,a4paper]{mwart}

\usepackage[utf8]{inputenc}
\usepackage{polski}
\usepackage[polish]{babel}

\usepackage[left=2.5cm, right=2.5cm, top=2.5cm, bottom=2.5cm]{geometry}

\begin{document}
	
		\begin{titlepage}

	    \begin{center}
		    \small
		    Politechnika Wrocławska\\
		    Wydział Elektroniki\\
		    Podstawy Technik Mikroprocesorowych
	    \end{center}

	    \vspace{6cm}

		\noindent\rule{\linewidth}{0.4mm}
	    \begin{center}
			\LARGE \textsc{Mikroprocesorowy Sterownik Rotora Antenowego} %Microprocessor controller for antenna rotor
	    \end{center}
		\noindent\rule{\linewidth}{0.4mm}

		\vspace{0.5cm}

	    \begin{flushright}
			\begin{minipage}{6cm}
				\textit{\small Autor:}\\
				\normalsize \textsc{Paweł Szwagierek}
			\end{minipage}

			\vspace{3cm}
			{\small Prowadzący:}\\
			dr inż. Jerzy Greblicki
	    \end{flushright}

	    \vspace*{\stretch{6}}

	    \begin{center}
		    \today
	    \end{center}	
    \end{titlepage}


	\title{}
	\author{Paweł Szwagierek\\
	Wydział Elektroniki,\\
	Politechnika Wrocławska,\\
	Nr Albumu: 184763\\
	\texttt{184763@student.pwr.wroc.pl}
	}
	\date{\today}
	\maketitle
	\tableofcontents
	\clearpage

	\section{Wstęp}
	***Celem ***

	\section{Potrzebne oprogramowanie}
	***fffisdasd***

	Dokumentacja przygotowana w \LaTeXe

	\section{Przygotowanie projektu w programie Keil}
	Pierwszym ważnym krokiem jest stworzenie kompilującego się pustego projektu zawierającego wszystkie potrzebne biblioteki. Po uruchomieniu programu zobaczymy puste okna.

	\section{Wnioski oraz podsumowanie}
	***WWW***
	\section{Bibliografia}
	\begin{thebibliography}{9}

	%The \bibitem is to start a new reference. Ensure that the cite_key is
	%unique. You don't need to put each element on a new line, but I did
	%simply for readability.

	\bibitem{lamport94}
	Leslie Lamport,
	\emph{\LaTeX: A Document Preparation System}.
	Addison Wesley, Massachusetts,
	2nd Edition,
	1994.

	\end{thebibliography} %Must end the environment

\end{document} %End of document.