\documentclass[12pt,a4paper]{mwart}

\usepackage[utf8]{inputenc}
\usepackage{polski}
\usepackage[polish]{babel}

\usepackage[left=2.5cm, right=2.5cm, top=2.5cm, bottom=2.5cm]{geometry}

\begin{document}
	
		\begin{titlepage}

	    \begin{center}
		    \small
		    Politechnika Wrocławska\\
		    Wydział Elektroniki\\
		    Podstawy Technik Mikroprocesorowych
	    \end{center}

	    \vspace{6cm}

		\noindent\rule{\linewidth}{0.4mm}
	    \begin{center}
			\LARGE \textsc{Mikroprocesorowy Sterownik Rotora Antenowego} %Microprocessor controller for antenna rotor
	    \end{center}
		\noindent\rule{\linewidth}{0.4mm}

		\vspace{0.5cm}

	    \begin{flushright}
			\begin{minipage}{6cm}
				\textit{\small Autor:}\\
				\normalsize \textsc{Paweł Szwagierek}
			\end{minipage}

			\vspace{3cm}
			{\small Prowadzący:}\\
			dr inż. Jerzy Greblicki
	    \end{flushright}

	    \vspace*{\stretch{6}}

	    \begin{center}
		    \today
	    \end{center}	
    \end{titlepage}


	\section{Wstęp}
	Celem projektu jest zaprojektowanie sterownika rotora antenowego dla zastosowań krótkofalarskich. Sterownik mikroprocesorowy powinien umieć skierować antenę w ustalonym kierunku według podanych współrzędnych horyzontalnych (Azymut i Elewacja).

	\section{Potrzebne oprogramowanie}
	***fffisdasd***

	Dokumentacja przygotowana w \LaTeXe

	\section{Przygotowanie projektu w programie Keil}
	Pierwszym ważnym krokiem jest stworzenie kompilującego się pustego projektu zawierającego wszystkie potrzebne biblioteki. Po uruchomieniu programu zobaczymy puste okna.
	
	\section{Wnioski oraz podsumowanie}
	***WWW***

\end{document} %End of document.